\documentclass{exam}
\usepackage[utf8]{inputenc}

\usepackage[a4paper, total={7.25in, 10.25in}]{geometry}
\usepackage{listings}
\usepackage{color}
\usepackage{amsfonts}
\usepackage{amsmath}

\usepackage[czech]{babel}

\footer{}{\thepage}{}
\pointpoints{bod}{bodů}

\begin{document}

{\large 2022/23 - Zápočtový test}

\begin{questions}
\question[6] \(U = \langle (0,1,-1,2),(1,1,-1,1)\rangle\), \(V = \langle (-1,-1,2,0),(2,1,-1,0),(1,1,0,2) \rangle\), \(v = (-1, 1, -1, 3)\). \(U, V \subset \mathbb{R}^4 \)

\begin{parts}
    \part[1.5] Určete dimenzi \(U + V\) a \(U \cap V\).
    \part[1.5] Určete nějakou bázi \(U \cap V\).
    \part[1.5] Určete, zda \(v \in U\). Pokud ano, určete souřadnice v \(\underline{u}\).
    \part[1.5] Najděte nějakou ortonormální bázi U (a souřadnice \(v\) v této bázi).
\end{parts}

\newpage

\question[6] \( A = \begin{pmatrix}
    1 & 0 & 0\\
    0 & 1 & 0\\
    0 & -1 & 0
\end{pmatrix} \).

\begin{parts}
    \part[1] Určete \(|A|\) a \(A^{-1}\).
    \part[1] Určete jádro a obraz zobrazení \(f: \mathbb{R}^3 \to \mathbb{R}^3: x \mapsto Ax \).
    \part[2] Určete vlastní čísla a vektory.
    \part[2] Najděte matici zobrazení \(f\) v bázi vlastních vektorů. Co je to za zobrazení?
\end{parts}
\end{questions}

\end{document}
