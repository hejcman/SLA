\newpage
{\large 2021 - 1}

\begin{questions}

\question Navrhněte dva vektorové prostory z \(\mathbb{R}^3\) tak aby platilo \(\texttt{dim} V = \texttt{dim} W = \texttt{dim}(V + W)\).

\question Rozhodněte, zda matice definována jako \(\begin{pmatrix}
    a & 1\\
    0 & a
\end{pmatrix}\) tvoří spolu s operací násobení grupu.

\question Definujte Dedekindův řez.

\question Definujte matici řádu 2 symetrické bilineární formy, která je negativně semidefinitní a zároveň není negativně definitní.

\question Platí, že stopa \(A^T + B\) je rovná stopě \(A + B^T\)?

\question Napište všechny matice řádu \(\mathbb{R}^1\) pro které platí, že jsou si zároveň inverzní maticí.

\end{questions}

\hrule

\begin{questions}

{\color{gray}

\question \(V = (0,0,0), W = (0,0,0)\)

\question Ne, chybí neutrální prvek.

\question

\question

\question Ano, transponování nemění prvky na diagonále.

\question \((-1), (1)\).

}

\end{questions}

\newpage
{\large 2021 - 2}

\begin{questions}

\question Dokažte, že každá čtvercová matice je vyjádřitelná jako součet symetrické a antisymetrické matice.

\question Uveďte příklad na čtvercovou matici 3. řádu, která není diagonální a má jen 1 vlastní hodnotu, a to 0.

\question Dokažte, že jádro homomorfismu \(\gamma: U \rightarrow V\) je podprostor \(U\).

\question Dokažte, že relace \(\Delta\) je ekvivalence pokud platí, že \(x \Delta y \Leftrightarrow x \sim y \wedge y \sim x\) kde operace \(\sim\) je reflexivní i tranzitivní.

\question Je \(\mathbb{Z} \rightarrow \mathbb{F}_2^n\) surjektivní zobrazení?

\question Napište 3 různé lineární rovnice se 4 neznámými tak, aby soustava byla neřešitelná.

\end{questions}

\hrule

\newpage
{\large 2021 - 3}

\begin{questions}

\question Napište definici uspořádaného pole.

\question Napište definici lineární kombinace libovolných vektorů.

\question Napište důkaz toho, že jádro homomorfismu je triviální pokud je Homomorfizmus injektivní.

\question \(\mathbb{Q}^*\) jsou nenulové racionální čísla s operací děleno. Jedná se o grupu?

\question Napište kolmý vektor na vektory \((2, -5, 1), (7, -12, 2)\).

\question Platí, že všechny antisymetrické matice 2. řádu mají jen imaginární vlastní hodnoty?

\end{questions}

\hrule
