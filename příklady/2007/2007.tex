
\newpage
{\large 2007 - 1}

\begin{questions}

\question Je dáno zobrazení \(f: Z \times N \rightarrow Q, f([m,n]) = \frac{m}{n}\). Určete, zda je dané zobrazení injekce, surjekce nebo bijekce.
\vspace{\stretch{1}}

\question Určete všechny matice \(X\) pro něž platí \(AX = O = XA\), kde \(A = \begin{pmatrix}
    3 & 4 & 2\\
    -2 & -1 & -1\\
    1 & 3 & 1\\
\end{pmatrix}\).
\vspace{\stretch{1}}

\newpage
\question Řešte systém rovnic s parametrem \(a\): \begin{center}
    \(\begin{matrix}
         x_1 & - &  x_2 & - &  x_3 & - &  3x_4 & = & 2 \\
        4x_1 & - & 2x_2 & + & 3x_3 & + &  7x_4 & = & 1 \\
         x_1 & - & 3x_2 & - & 8x_3 & - & 22x_4 & = & 9 \\
         x_1 & - & 3x_2 & + & 7x_3 & + & 17x_4 & = & a \\
    \end{matrix}\)
\end{center}
\vspace{\stretch{1}}

\question Ukažte, že přímky \(p: \left\{
    \begin{array}{l}
        3x + 2y - z + 1 = 0\\
        x + y - 3z + 3 = 0
    \end{array}
\right.\) a \(q: \left\{
    \begin{array}{l}
        5x + y + 4z - 3 = 0\\
        2x + y + 2z - 2 = 0
    \end{array}
\right.\) leží v téže rovině a napište její rovnici.
\vspace{\stretch{1}}

\newpage
\question Najděte rovnici kulové plochy, která prochází body \(A = [2, -4,2], B = [-4, 8, 2], C = [5, -1, 14], D = [-7, -4, 5]\).
\vspace{\stretch{1}}

\end{questions}

\newpage
{\large 2007 - 2}

\begin{questions}

\question Grupoid \((R, \cdot)\), kde \(x \cdot y = (x+y)(1+xy)\), \(x, y \in \mathbb{R}\). Ověřte jestli je asociativní, komutativní, jestli existuje jednička grupoidu. Jestliže ano, určete inverzní prvky.
\vspace{\stretch{1}}

\question Je dána matice \(A = \begin{pmatrix}
    1 & 2\\
    2 & 1\\
\end{pmatrix}\). Vypočtěte matice \(B = (I+A)(I-A)^{-1}\), \(C = (I-A)^{-1}(I+A)\), pokud existují.
\vspace{\stretch{1}}

\newpage

\question Čemu se musí rovnat \(\lambda\), aby systém rovnic měl řešení? Určete všechna jeho řešení pro tuto hodnotu \(\lambda\). \begin{center}
    \(\begin{matrix}
        2x_1 & - &  x_2 & + &  x_3 & + &   x_4 & = & 1 \\
         x_1 & + & 2x_2 & - &  x_3 & + &  4x_4 & = & 2 \\
         x_1 & + & 7x_2 & - & 4x_3 & + & 11x_4 & = & \lambda
    \end{matrix}\)
\end{center}
\vspace{\stretch{1}}

\question Najděte přímku \(q\), která je kolmým průmětem přímky \(p: \left\{
    \begin{array}{l}
        x = 2 + 7t \\
        y = -1 - 4t \\
        z = 1 - 6t
    \end{array}
\right.\), \(t \in \mathbb{R}\) na rovinu \(\rho:  x - 2y - z + 8 = 0\).
\vspace{\stretch{1}}

\newpage
\question Najděte rovnice rotačních kuželových ploch, jejichž osa je osa \(z\), jež procházejí bodem \(M = [6,8,-3]\) a tvořící přímky svírají s osou úhel \(\frac{\pi}{4}\).

\end{questions}
