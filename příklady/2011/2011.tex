
\newpage
{\large 2011 - 1}

\begin{questions}

\question Zjistěte, zda ve vektorovém prostoru \(\mathbb{V} = (\mathbb{R}^3, +)\) lze definovat skalární součin vztahem \( \langle \underline{u}, \underline{v} \rangle = 3u_1v_1 - u_2v_1 + 2u_2v_2 + u_1v_3 + u_3v_1 + u_3v_3 \), kde \(\underline{u} = (u_1, u_2, u_3)\), \(\underline{v} = (v_1, v_2, v_3)\).
\vspace{\stretch{1}}

\question Jsou-li \(A, B\) regulární zaměnitelné (?) matice, ukažte, že jsou zaměnitelné také matice \(A^{-1}, B, A, B^{-1}, A^{-1}, B^{-1}\).
\vspace{\stretch{1}}

\newpage

\question Řešte systém lineárních rovnic \(\begin{matrix}
    x & - & by & = & 1\\
    x & + & ay & = & 3
\end{matrix}\) v závislosti na parametrech \(a, b \in \mathbb{R}\).
\vspace{\stretch{1}}

\question Zjistěte, zda přímky \(p_1: \left\{
    \begin{array}{l}
        2x+y-z =0\\
        x-3y+2z-14 =0
    \end{array}
\right.\) a \(p_2: \left\{
    \begin{array}{l}
        x+5y-6z+34 = 0\\
        6x-2y-z-9=0
    \end{array}
\right.\) jsou rovnoběžné nebo různoběžné\footnote{Na fotce se rovněž objevila i další nečitelná "-běžnost".}.
\vspace{\stretch{1}}

\newpage

\question Ke kulové ploše \(\mathcal{S}: x^2 + y^2 + z^2 - 8x - 4z - 205 = 0\) veďte tečné roviny rovnoběžné s rovinou \(\rho: 10x - 11y - 2z + 3 = 0\).
\end{questions}

\newpage
{\large 2011 - 2}
