
\newpage
{\large 2021 - 0} (Řešeno společně na tabuli na poslední přednášce.)

\begin{questions}

\question V \(\mathbb{R}^6\) jsou vektorové podprostory \(V, W\). \(V: v_1 = v_2 = v_3 = v_4\), \(W: w_1 + w_2 + w_3 = w_4 + 3w_5 + 5w_6\). Najděte ortonormální bázi \(V \cap W\).
\vspace{\stretch{1}}

\question V \(\mathbb{R}: \begin{matrix}
    3a & + & 4b & + &            5c & + & 6d & = & 1\\
    2a & + & 5b & + &            7c & + & 9d & = & 2\\
    7a & + & 7b & + & \frac{29}{3}c & + & 2d & = & 2\\
    5a & + & 9b & + &           14c & + & 6d & = & 3
\end{matrix}\). Najděte \(a-b\).
\vspace{\stretch{1}}

\newpage

\question \(U = \begin{pmatrix}
    1 &  0 & 1 &  i & 1\\
    2 & 4i & i &  i & 0\\
    0 &  1 & w & 2i & 0\\
    1 &  0 & 1 &  i & 0\\
    7 &  0 & 0 &  i & 0
\end{pmatrix}\). \(|U| = -5i\), spočtěte \(u\).
\vspace{\stretch{1}}

\question Homomorfizmus \(\gamma: \mathbb{R}^4 \rightarrow \mathbb{R}^2\). \(\gamma((v_1, v_2, v_3, v_4)) = (v_1+v_2+v_3, 8v_3+9v_4)\).

\(B = ((1,1,1,1),(1,1,1,0),(1,1,0,0),(1,0,0,0))\), \(\overline{B} = ((2,11),(1,0))\). Najděte matici homomorfismu v bázích \(B, \overline{B}\).
\vspace{\stretch{1}}

\newpage

\question V \(\mathbb{F}_{14}\) řešte \(\begin{pmatrix}
    1 & 2\\
    1 & 3\\
    1 & 4
\end{pmatrix} \times X = \begin{pmatrix}
    0 & 0 & 2\\
    0 & 3 & 4\\
    4 & 4 & 4
\end{pmatrix}\).
\vspace{\stretch{1}}

\question Určete a načrtněte \(4x^2 + 4z^2 + 36x - 44z + 202 = 0\).
\vspace{\stretch{1}}

\end{questions}
