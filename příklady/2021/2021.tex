
\newpage
{\large 2021 - 1}

\begin{questions}

\question Vektorový podprostor \(W\) je podprostorem \(\mathbb{R}^5\) a platí \(\left\{
    \begin{array}{l}
        w_1 + w_2 + w_3 = 3w_4 + 6w_5\\
        w_1 + w_3 + w_5 = -2w_4
    \end{array}
\right.\). Napište ortogonální bázi.
\vspace{\stretch{1}}

\question Řešte reálnou soustavu lineárních rovnic v \(\mathbb{R}\): \(\begin{matrix}
    4a & + &   b & + & 2c & + &  d & = & -1\\
    6a & + &  3b & + & 4c & + & 2d & = & -4\\
    6a & + &  7b & - &  c & - &  d & = & 11\\
    8a & + & 11b & + & 2c & + & 2d & = & -2\\
\end{matrix}\).
\vspace{\stretch{1}}

\newpage
\question \(\rho\) prochází \([5,0,2], [6,-2,4], [3,8,1]\), \(p\) prochází \(D=[1,1,1]\) a je kolmá na \(\rho\). \(E\) leží na \(p\), vzdálenost od \(\rho\) je \(12\sqrt{2}\). \(C, E\) prochází přímka \(q\), určete odchylku \(p\) a \(q\).
\vspace{\stretch{1}}

\question Nad \(\mathbb{F}_3\) máme matici \(M = \begin{pmatrix}
    1 & 1 & 0 & 0 & 2 & 2\\
    1 & a & 0 & 0 & a & 1\\
    1 & 0 & 1 & 0 & 1 & 2\\
    0 & 2 & 0 & 2 & 0 & 2\\
    1 & 1 & 2 & 1 & 1 & 2\\
    a & 0 & 1 & 2 & 1 & 0\\
\end{pmatrix}\). Pro jaké hodnoty \(a\) je matice \(M\) regulární?
\vspace{\stretch{1}}

\newpage
\question Najděte matice homomorfismu \(\mathbb{C} \rightarrow \mathbb{C}\), \(\mathcal{B} = ((1,1,1),(i,i,0),(i,0,0))\), \(\overline{\mathcal{B}} = ((1,1,i), (1,i,0), (1,0,0)\), \(\varphi(z_1, z_2, z_3) = (z_1-z_2, 0, iz_2-z_3))\).
\vspace{\stretch{1}}

\question Vlastní vektory matice \(\begin{pmatrix}
    1 & 1 & 0\\
    1 & 1 & 0\\
    2 & 2 & 2
\end{pmatrix}\) generují podprostor \(V \subseteq \mathbb{R}^3\). Najděte bázi \(W: W \oplus V = \mathbb{R}^3\).\footnote{\(\oplus\) značí nulový průnik.}
\vspace{\stretch{1}}

\end{questions}

%%%%%%%%%%%%%%%%%%%%%%%%%%%%%%%%%%%%%%%%%%%%%%%%%%%%%%%%%%%%%%%%%%%%%%%%%%%%%%%

\newpage
{\large 2021 - 2}

\begin{questions}

\question V \(\mathbb{R}^6\) máme vektorové pole \(V, W\), kde \(V = \{ (a, b, a, 2a, 3a, 4a) | a, b \in \mathbb{R} \}\). \(W\) generuje \(\overrightarrow{w_1} = (\frac{1}{4}, \frac{1}{4}, \frac{1}{4}, \frac{1}{2}, \frac{3}{4}, 1)\), \(\overrightarrow{w_2} = (0, 1, 0, 0, 0, 1)\), a \(\overrightarrow{w_3} = (\frac{1}{4}, 0, \frac{1}{4}, \frac{1}{2}, \frac{3}{4}, \frac{3}{4})\). Najděte ortogonální bází \(V + W\).
\vspace{\stretch{1}}

\question Pro jaké \(\alpha\) je reálná matice \(M\) regulární? \(\begin{pmatrix}
    1        &        1 & 1 & -8\alpha\\
    1        & \alpha^2 & 1 & \alpha^2\\
    1        &   \alpha & 1 &  -\alpha\\
    \alpha^4 &        1 & 0 &        1\\
\end{pmatrix}\)
\vspace{\stretch{1}}

\newpage
\question V \(\mathbb{C}\) řešte \(\begin{matrix}
     (1+i)a & + &  (2+i)b & + &  (3+i)c & = & -3i\\
     (1+i)a & + & (1+2i)b & + & (1+3i)c & = & 1\\
    (3-2i)a & + & (1-3i)b & + &  (2-i)c & = & 0\\
\end{matrix}\)
\vspace{\stretch{1}}

\question Najděte matici přechodu v \(\mathbb{F}_7^3\) z báze \(\beta = ((1,0,5),(1,0,4),(0,1,0))\) do báze \(\gamma = ((1,0,2),(1,0,3),(0,4,4))\).
\vspace{\stretch{1}}

\newpage
\question Přímka \(p\) prochází body \([1,1,0], [2,3,3]\). Přímka \(q\) prochází body \([4,0,-1], [8,2,2]\). Ověřte, že nejsou mimoběžky.
\vspace{\stretch{1}}

\question Najděte vlastní vektor matice \(M\), který neleží v žádné z souřadnicových (?) rovin. \(M = \begin{pmatrix}
    1 & 1 & 1\\
    1 & 1 & 2\\
    2 & 2 & 1
\end{pmatrix}\).
\vspace{\stretch{1}}

\end{questions}

%%%%%%%%%%%%%%%%%%%%%%%%%%%%%%%%%%%%%%%%%%%%%%%%%%%%%%%%%%%%%%%%%%%%%%%%%%%%%%%

\newpage
{\large 2021 - 3}

\begin{questions}

\question \(V, W\) jsou vektorové podprostory \(\mathbb{R}^6\). \(V\) generuje \((1,2,3,-1,8,6),(1,8,5,-7,23,27),(2,-2,4,4,1,-9)\). \(W\) generuje \((-3,6,-5,-9,6,24),(1,-10,-1,11,-22,-36),(7,2,17,5,26,0)\). Najděte ortogonální bázi \(V \cap W\).
\vspace{\stretch{1}}

\question V \(\mathbb{R}\) řešte soustavu \(\begin{matrix}
     4a & + &  b & + &   c & + &   d & = & -6\\
     7a & + & 2b & - &  2c & - &   d & = & 5\\
     8a & - & 2b & + & 14c & + & 13d & = & 7\\
    28a & + &  b & + &   c & + &  7d & = & -12\\
\end{matrix}\)
\vspace{\stretch{1}}

\newpage
\question Mějme matici \(M = \begin{pmatrix}
                1 &             1 &             1 & a\sqrt{3} \\
                1 &             1 & \sqrt{3}(1+a) & a\sqrt{3} \\
                1 & \sqrt{3}(2+a) & \sqrt{3}(1+a) & a\sqrt{3} \\
    \sqrt{3}(3+a) & \sqrt{3}(2+a) & \sqrt{3}(1+a) & a\sqrt{3} \\
\end{pmatrix}\)
\vspace{\stretch{1}}. Vypočítejte \(|M|\) pro \(a=\frac{\sqrt{3}}{3}\) a \(a = \frac{\sqrt{3}}{3}-1\).

\question V \(\mathbb{F}_{11}\) řešte \(X * \begin{pmatrix}
    1 & 2\\
    3 & 6\\
    4 & 8\\
    5 & 10
\end{pmatrix} = \begin{pmatrix}
    3 & 6\\
    9 & 7\\
    4 & 8
\end{pmatrix}\).
\vspace{\stretch{1}}

\newpage
\question Vzájemná poloha roviny \(\rho\) a přímky \(p\) jsou \(\rho: A=[3,0,2], B=[4,-1,5], C=[5,-3,-1]\), \(p: K=[2,-6,-64], \overrightarrow{u}=(0,15,101)\). Pokud jsou rovnoběžné, určete průnik, jinak určete vzdálenost.
\vspace{\stretch{1}}

\question Najděte průměr vlastních hodnot matice \(M=\begin{pmatrix}
    0 & 0 & 5\\
    2 & 1 & 0\\
    2 & 0 & -4\\
\end{pmatrix}\).
\vspace{\stretch{1}}
\end{questions}


%%%%%%%%%%%%%%%%%%%%%%%%%%%%%%%%%%%%%%%%%%%%%%%%%%%%%%%%%%%%%%%%%%%%%%%%%%%%%%%

\newpage
{\large 2021 - 4}

\begin{questions}

\question V \(\mathbb{R}^4\) jsou vektorové podprostory \(V, W\), kde \(V\) generuje vektory \((1,1,-2,3), (4,-1, 2,-2)\), a \(W\) generuje vektory \((3,2,-5,0),(1,-4,8,-11)\). Najděte ortogonální bázi \(V+W\).
\vspace{\stretch{1}}

\question V \(\mathbb{R}\) jsou dány následující rovnice. Spočtěte \(b-c\). \begin{center}
    \(\begin{matrix}
        3a & + & 4b & + &  5c & + &   6d & = & 10\\
        3a & + & 5b & + &  7c & + &   9d & = & -33\\
        5a & + & 7b & + & 10c & + & 272d & = & 28\\
        5a & + & 9b & + & 13c & + & 276d & = & \frac{8}{3}
    \end{matrix}\)
\end{center}
\vspace{\stretch{1}}

\newpage
\question \(U=M^4, V=M^3, U=\begin{pmatrix}
              1 & 1+2\sqrt{7} & 0 &            0 & 0\\
              1 &           0 & 2 &            0 & 0\\
    3-2\sqrt{7} &           3 & 3 &            1 & 1\\
              2 &           1 & 3 &            1 & 0\\
              0 &           0 & 2 & 3+10\sqrt{7} & 1
\end{pmatrix}\)
\vspace{\stretch{1}}

\question Přímka \(p\) prochází body \(A=[7,2,2], B=[4,-1,1]\). Přímka \(q\) prochází body \(C=[8,0,-1], D=[2,12,-3]\). Ověřte, že jde o různoběžky a spočtěte jejich ?.
\vspace{\stretch{1}}

\newpage
\question V \(\mathbb{F}_7\) je matice \(S = \begin{pmatrix}
    3 & 0\\
    1 & 2
\end{pmatrix}\). Spočtěte \(S^{12}\).
\vspace{\stretch{1}}

\question \(U = \begin{pmatrix}
    1 & 0 & 1\\
    1 & 0 & 1\\
    1 & 2 & 0
\end{pmatrix}\). Napište dva lineárně nezávislé vlastní vektory matice \(U\) s třetí souřadnicí rovnou 6.
\vspace{\stretch{1}}
\end{questions}
