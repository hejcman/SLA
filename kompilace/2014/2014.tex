
{\large 2014 - 1}

\begin{questions}

\question Napište obecnou rovnici přímky \(p: x=2-t, y=1+3t\), \(t \in \mathbb{R}\).
\vspace{\stretch{1}}

\question Napište parametrické vyjádření přímky určené v \(\mathbb{R}^3\) rovnicemi

\begin{center}
    \(\begin{matrix}
        x  & - & 2y & + & z & = & 2\\
        2x & + & y  & - & z & = & 5
    \end{matrix}\)
\end{center}
\vspace{\stretch{1}}

\newpage

\question Nechť \(V = \mathbb{R}^\mathbb{R}\), \((f+g)(x) = f(x)+g(x)\), \((r \cdot f)(x)=r \cdot f(x)\), pro každé \(f, g \in \mathbb{R}^\mathbb{R}, r, x \in \mathbb{R}\).

\begin{parts}
    \part Ukažte, že \((V, +, \cdot)\) je vektorový prostor.
    \part Rozhodněte, zda množina \(U = \{ f \in V | f(0) = f(1) \}\) je podprostorem \((V, +, \cdot)\).
\end{parts}
\vspace{\stretch{1}}

\question Zjistěte, zda ve vektorovém prostoru \(\mathbb{P}_3, +, \cdot\) vektory \(\textbf{u} = x^3+x\), \(\textbf{v} x^2+1, \textbf{w} = x^3-x^2+x+1\)

\begin{parts}
    \part jsou lineárně nezávislé;
    \part tvoří bázi.
\end{parts}
\vspace{\stretch{1}}

\newpage

\question Nechť \(A\) je diagonální matice řádu \(n\). Formulujte pravidlo pro výpočet součinů \(X \cdot A\), \(A \cdot Y\), kde \(X, Y\) jsou libovolné matice typu \(m \times n\), \(n \times m\).
\vspace{\stretch{1}}

\question Vypočtěte determinant \(\begin{vmatrix}
    2 & 0 & 1 & 3\\
    0 & 1 & 3 & 2\\
    1 & 3 & 2 & 0\\
    3 & 2 & 0 & 1
\end{vmatrix}\)
\vspace{\stretch{1}}

\end{questions}

\newpage
{\large 2014 - 2}

\begin{questions}

\question Pomocí inverzní matice určete matici \(X\), pro kterou platí
\(
    \begin{pmatrix}
        1 & 2 & -3\\
        3 & 2 & -4\\
        2 & -1 & 0
    \end{pmatrix}
    \cdot X =
    \begin{pmatrix}
        1 & -3 & 0\\
        10 & 2 & 7\\
        10 & 7 & 8
    \end{pmatrix}
\)
\vspace{\stretch{1}}

\question Cramerovým pravidlem řešte následující systém lineárních rovnic:
\begin{center}
    \(\begin{matrix}
        5x & + & y & - & z  & = & -7\\
        2x & - & y & - & 2x & = & 6\\
        3x &   &   & + & 2z & = & -7
    \end{matrix}\)
\end{center}
\vspace{\stretch{1}}

\newpage

\question Řešte systém lineárních rovnic, víte-li, že má řešení \([1, 8, 13, 0, -34]\):

\begin{center}
    \(\begin{matrix}
        6x_1 & + & 4x_2 & + & 5x_3 & + & 2x_4 & + & 3x_5 & = & 1 \\
        3x_1 & + & 2x_2 & + & 4x_3 & + & x_4  & + & 2x_5 & = & 3 \\
        3x_1 & + & 2x_2 & - & 2x_3 & + & x_4  &   &      & = & -7\\
        9x_1 & + & 6x_2 & + & x_3  & + & 3x_4 & + & 2x_5 & = & 2
    \end{matrix}\)
\end{center}
\vspace{\stretch{1}}

\question Určete matici \(\mathcal{P}_{\mathcal{D} \leftarrow \mathcal{B}}\), kde \(\mathcal{B}\) a \(\mathcal{D}\) jsou dané uspořádané báze vektorového prostoru \(V\):

\begin{center}
    \(V = \mathbb{P}_2\), \(\mathcal{B} = \{ x, 1+x, x^2 \}\), \(\mathcal{D} = \{ 2, x+3, x^2-1 \}\)
\end{center}

\vspace{\stretch{1}}

\newpage
\question Najděte charakteristický polynom, vlastní čísla a vlastní vektory matice \(\begin{pmatrix}
    7 & 0 & -4\\
    0 & 5 & 0\\
    5 & 0 & -2
\end{pmatrix}\)
\end{questions}
