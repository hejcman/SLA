
\newpage
{\large 2009 - 1}

\begin{questions}

\question Rozhodněte, zda \(f: \mathbb{R} \rightarrow \mathbb{R} \times \mathbb{R}\)f dané vztahem \(f(x) = [x+1,x-1]\) je injektivní, subjektivní nebo bijektivní.
\vspace{\stretch{1}}

\question Nechť \(A\) je čtvercová matice, \(k \in \mathbb{R}, k \neq 0\) a nechť \(A^2 = kA\) = kA. Ukažte, že matice \(A\) je regulární, právě když \(A = k * I\).
\vspace{\stretch{1}}

\newpage

\question Řešte rovnice v závislosti na parametru \(\lambda\): \\\begin{center}
    \(\begin{matrix}
        2x_1 & + & 5x_2  & + & x_3  & + & 3x_4        & = & 2\\
        4x_1 & + & 6x_2  & + & 3x_3 & + & 5x_4        & = & 4\\
        4x_1 & + & 14x_2 & + & x_3  & + & 7x_4        & = & 4\\
        2x_1 & - & 3x_2  & + & 3x_3 & + & \lambda x_4 & = & 7\\
    \end{matrix}\)
\end{center}
\vspace{\stretch{1}}

\question Najděte roviny symetrie různoběžných rovin \(\rho_1: 2x+5y-5z+16=0\) a \(\rho_2: 2x - 7y -z + 8 = 0\).
\vspace{\stretch{1}}

\newpage

\question Určete rovnici kuželové plochy \(\mathcal{S}\) s vrcholem \(V = [-1,1,8]\) a řídící křivkou \(L: x^2 + y^2 -4 = 0, z-4=0\).
\vspace{\stretch{1}}

\end{questions}

\newpage
{\large 2009 - 2}

\begin{questions}

\question Určete vlastní čísla a vlastní prostory matice \(A = \begin{pmatrix}
    3  & 5  & 3\\
    -4 & -9 & -6\\
    6  & 15 & 10\\
\end{pmatrix}\)
\vspace{\stretch{1}}

\question Nechť \(A,B\) jsou diagonální matice téhož řádu. Ukažte, že \(AB\) je též diagonální a že matice \(A,B\) jsou zaměnitelné.
\vspace{\stretch{1}}

\newpage

\question Řešte systém lineárních rovnic: \begin{center}
    \(\begin{matrix}
        3x_1 & + & 2x_2 & + & 2x_3 & + & 2x_4 & = & 2 \\
        2x_1 & + & 3x_2 & + & 2x_3 & + & 5x_4 & = & 3 \\
        9x_1 & + &  x_2 & + & 4x_3 & - & 5x_4 & = & 1 \\
        2x_1 & + & 2x_2 & + & 3x_3 & + & 4x_4 & = & 5 \\
        7x_1 & + &  x_2 & + & 6x_3 & - & 1x_4 & = & 7 \\
    \end{matrix}\)
\end{center}
\vspace{\stretch{1}}

\question Určete rovnice dvou navzájem kolmých rovin \(\delta_1, \delta_2\) procházejících přímkou \(p: \left\{
    \begin{array}{l}
        3x + y - z -4 = 0\\
        x - 2y + 4z - 2 = 0
    \end{array}
\right.\) z nichž první prochází bodem \(A=[ 2, -3, 4 ]\).
\vspace{\stretch{1}}

\newpage

\question Určete rovnice povrchových přímek hyperbolického paraboloidu \(\frac{x^2}{16}-\frac{y^2}{4}-z = 0\), které jsou rovnoběžné s rovinou \(\delta: 3x + 2y - 4z = 0\). Určete také jejich průsečík a rovinu, která je jimi určena.
\end{questions}
